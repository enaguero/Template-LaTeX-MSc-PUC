The aim of this project is create a platform which allows to estimate and predict the number of students will take a course in The Engineering School in the next academic period.
This need borns for the implementation of the new acamedic plan created in 2013, this program is divided into 2 formative phases: Bachelor of Engineering, during the first 4 years, and
follows for an UC Professional Title or another academic degree. The flexibility that allows this structure has as consecuence that one specific course can be founded in the acamedic
path of very diverse students, it brings a really high variability in the number of students whom take the course.\

In order to manage the courses demand was created a web plataform using the framework Ruby On Rails and R ,the language and environment for statistical computing and graphics. These
technologies allow to face in a modular way the challenges to create predictives models and manage the information of them through the web platform. In the the requirements gathering process
was used the methodology User Stories and for the design of predictive models was choosen Cross Industry Standard Process for Data Mining (CRISP-DM). In general, the web platform manage
all the information used by the predictive models and allow load new data, update and/or delete the current information, at the same time, it allows to operate the predictive models
for each course and visualize the results obtained.\

To conclude, the results of the predictions are presented based on the data provided by the \textit{Dirección de Pregrado}, as well as a list of solutions to the problems that exits for
the lack of data given that there is still no generation of students who have completed the first cycle.\
