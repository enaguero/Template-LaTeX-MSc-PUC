El objetivo del presente trabajo es crear una  plataforma  que permita estimar y predecir la cantidad de alumnos que tomarán un curso en la Escuela de Ingeniería en el siguiente periodo
académico. Esto nace gracias al nuevo plan de estudios implementado por la Escuela  de Ingeniería en el 2013, éste se encuentra dividido en 2 ciclos formativos:
Licenciatura en Ciencias de la Ingeniería, correspondiente a los primeros 4 años, y articulación con un Título Profesional UC u otros grados académicos.
La flexibilidad de esquema tiene como consecuencia que un curso puede estar en el camino académico de alumnos muy diversos, lo que se traduce en alta variabilidad en el número de alumnos
que toman el curso.

Para gestionar la demanda de cursos dado los ciclos formativos en la Escuela de Ingeniería se diseñó una plataforma web utilizando el framework Ruby On Rails en conjunto con el entorno
estadístico R. Este stack de tecnologías permite abordar de forma modular los desafíos a nivel de información para la creación de modelos predictivos y gestión de información a través de
la plataforma web. Se enfrentó el proceso de levantamiento de requerimientos con la metodología de Relatos de Usuario y para el diseño de modelos predictivos se escogió la metodología
Cross Industry Standard Process for Data Mining (CRISP-DM). A modo general, la plataforma web permite gestionar la información presente (cargar nuevos datos , actualizar y/o eliminación
la información ya existente), gestionar los modelos predictivos para cada curso y visualizar lo resultados obtenidos.

Para concluir, se presentan los resultados de las predicciones en base a los datos proporcionados por Dirección de Pregrado, además de un listado de soluciones a los problemas que se
presentan por la falta de datos dado que aún no existe una generación de alumnos que haya finalizado el primer ciclo.
