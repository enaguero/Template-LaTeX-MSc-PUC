\begin{enumerate}
  \item \label{refactoring} \textbf{\textit{Refactoring}}\mbox{}\\ Es el proceso de reconstrucción de código existente sin cambiar su comportamiento. Mejora los aspectos non funcionales del \textit{software}, por ejemplo, rendimiento.
  \item \label{tdd} \textbf{\textit{Test Driven Development}}\mbox{}\\ Proceso de desarrollo de software el cual se basa en la repetición de un ciclo de tres pasos. Primero, el desarrollador escribe un test automático que define el comportamiento deseado del software; segundo, se escribe la mínima cantida de código que hace el test pasar y ,finalmente, se hace un proceso de \textit{refactor} del código.
  \item \label{pair_programming} \textbf{Programación a pares}\mbox{}\\ Proceso de programación donde dos programadores participan en un esfuerzo combinado de desarrollo en un sitio de trabajo. Cada miembro realiza una acción que el otro no está haciendo actualmente: Mientras que uno codifica las pruebas de unidades el otro piensa en la clase que satisfará la prueba, por ejemplo.
  \item \label{continuous_integration} \textbf{Integración continua}\mbox{}\\ Práctica de desarrollo de software mediante la cual los desarrolladores combinan los cambios en el código en un repositorio central de forma periódica, tras lo cual se ejecutan versiones y pruebas automáticas. La integración continua se refiere en su mayoría a la fase de creación o integración del proceso de publicación de software y conlleva un componente de automatización (p. ej., CI o servicio de versiones) y un componente cultural (p. ej., aprender a integrar con frecuencia). Los objetivos clave de la integración continua consisten en encontrar y arreglar errores con mayor rapidez, mejorar la calidad del software y reducir el tiempo que se tarda en validar y publicar nuevas actualizaciones de software \cite{aws}.
  \item \label{focus_group} \textbf{Focus Group}\mbox{}\\ Tipo de técnica de estudio empleada en las ciencias sociales y en trabajos comerciales que permite conocer y estudiar las opiniones y actitudes de un público determinado \cite{abc}.
  \item \label{epic} \textbf{Epic}\mbox{}\\ Agrupación de relatos de usuarios. También son considerados relatos de usuarios más grandes \cite{epic}.
  \item \label{mvp} \textbf{Minimal Viable Product} \mbox{} \\ Producto con suficientes características para satisfacer a los clientes iniciales, y proporcionar retroalimentación para el desarrollo futuro \cite{mvp}.
  \item \label{ontoligical} \textbf{Ingeniería Ontológica} \mbox{} \\ es un campo de las ciencias de la computación y ciencias de la información que estudia los métodos y metodologías para construir esquemas conceptuales (ontología): ésta corresponde a la representación formal de un grupo de conceptos dentro de un dominio y de las relaciones entre esos conceptos. Una representación a gran escala de conceptos abstractos como acciones, tiempo, objetos físicos y creencias podría ser un ejemplo de ingeniería ontológica \cite{ontology_engineering}.
\end{enumerate}
